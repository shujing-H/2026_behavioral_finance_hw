\documentclass[11pt]{article}

\usepackage[margin=1in]{geometry}
\usepackage{amsmath, amssymb}
\usepackage{booktabs}
\usepackage{graphicx}
\usepackage{subcaption}
\usepackage{hyperref}
\usepackage{xcolor}
\usepackage{listings}
\usepackage{microtype}

\hypersetup{
  colorlinks=true,
  linkcolor=blue,
  urlcolor=blue,
  citecolor=blue
}

\lstset{
  basicstyle=\ttfamily\small,
  breaklines=true,
  frame=single,
  rulecolor=\color{black!20},
  showstringspaces=false,
  columns=fullflexible
}

\title{Homework 1 Report: Momentum}
\author{Shujing Hou \quad 3040781643 \\ Wenyu Guo \quad 3040782423}
\date{\today}

\begin{document}
\maketitle
\sloppy

\section{Overview and data}
\textbf{Goal.} This report studies the persistence of momentum-based strategies using monthly returns for the top 1,000 U.S. stocks by market capitalization (1980--present) and the Fama--French 3 factors.

\textbf{Signal definition (12--1 momentum).}
At the end of month $t$, the tradable momentum signal for stock $i$ is the cumulative return from $t-12$ to $t-1$:
\[
\text{MOM}_{i,t}^{12\text{--}1} \;=\; \prod_{k=1}^{12} (1+r_{i,t-k}) - 1.
\]
In the implementation, we compute this product in a numerically stable way using log-returns:
\[
\prod_{k=1}^{12} (1+r_{i,t-k}) \;=\; \exp\!\left(\sum_{k=1}^{12}\log(1+r_{i,t-k})\right).
\]

\textbf{Factor scaling note.} The stock returns in \texttt{returns\_indiv.csv} are in decimals, while the Fama--French factor returns in \texttt{monthly\_ff3.csv} are in percentages. For regressions we convert portfolio returns to percentages so that excess returns and factor returns are on the same scale, matching the notebook.

\section{Question 1: Raw returns of 10 momentum-sorted portfolios}
\subsection{Method}
Each month $t$:
\begin{itemize}
  \item Compute $\text{MOM}^{12\text{--}1}_{i,t}$ for every stock.
  \item Sort stocks into 10 deciles based on the \emph{cross-sectional rank} of the signal within month $t$ (decile 1 = lowest momentum, decile 10 = highest momentum).
  \item Form portfolio returns using three weighting schemes (computed separately each month and rebalanced monthly):
  \begin{itemize}
    \item \textbf{Equal-weighted (EW)}: simple mean return within each decile.
    \item \textbf{Value-weighted (VW)}: weights proportional to market capitalization within each decile.
    \item \textbf{Rank-weighted (RW)}: weights proportional to the momentum rank within each decile (higher momentum $\Rightarrow$ larger weight).
  \end{itemize}
  \item Hold the decile portfolios for 1 month using the next-month return (\texttt{ret\_p1}).
\end{itemize}

\subsection{Results}
\begin{table}[h!]
\centering
\caption{Average monthly raw returns by momentum decile (in \%).}
\label{tab:q1_raw_returns}
\begin{tabular}{lrrr}
\toprule
Decile & EW (\%) & VW (\%) & RW (\%) \\
\midrule
1 & 0.607 & 0.606 & 0.750 \\
2 & 0.851 & 0.842 & 0.854 \\
3 & 0.914 & 0.909 & 0.909 \\
4 & 0.870 & 0.866 & 0.867 \\
5 & 0.972 & 0.960 & 0.973 \\
6 & 1.014 & 1.005 & 1.008 \\
7 & 1.017 & 1.014 & 1.015 \\
8 & 1.041 & 1.027 & 1.032 \\
9 & 1.091 & 1.083 & 1.082 \\
10 & 1.388 & 1.367 & 1.376 \\
\bottomrule
\end{tabular}

\end{table}

\begin{figure}[h!]
\centering
\begin{subfigure}{0.32\textwidth}
  \includegraphics[width=\linewidth]{hw1_report_assets/q1_cumret_ew.png}
  \caption{EW cumulative returns}
\end{subfigure}
\begin{subfigure}{0.32\textwidth}
  \includegraphics[width=\linewidth]{hw1_report_assets/q1_cumret_vw.png}
  \caption{VW cumulative returns}
\end{subfigure}
\begin{subfigure}{0.32\textwidth}
  \includegraphics[width=\linewidth]{hw1_report_assets/q1_cumret_rw.png}
  \caption{RW cumulative returns}
\end{subfigure}
\caption{Cumulative returns of momentum-sorted decile portfolios (dashed) and the long--short spread (black) under each weighting scheme.}
\label{fig:q1_cumret}
\end{figure}

\begin{figure}[h!]
\centering
\includegraphics[width=0.78\textwidth]{hw1_report_assets/q1_longshort_timeseries.png}
\caption{Long--short (decile 10 minus decile 1) portfolio returns over time for EW/VW/RW.}
\label{fig:q1_longshort_series}
\end{figure}

\subsection{Analysis}
The core diagnostic in a momentum sort is whether returns are \textbf{monotone} in the signal: higher momentum deciles should earn higher next-month returns on average.
Table~\ref{tab:q1_raw_returns} and Figure~\ref{fig:q1_cumret} show the decile return pattern and the implied long--short spread.

\textbf{What we learn from the weighting comparison.}

EW, VW, and RW do \emph{not} just rescale the same strategy; they change which stocks inside each decile drive the realized return:
\begin{itemize}
  \item \textbf{EW} puts relatively more weight on smaller names (within the top-1000 universe), so it is more sensitive to momentum in smaller firms.
  \item \textbf{VW} concentrates on the largest firms. If momentum is weaker among mega-caps, VW spreads are typically smaller; if momentum is strong among large firms, VW and EW will look similar.
  \item \textbf{RW} intentionally emphasizes the \emph{extremes} within a decile by giving higher-signal stocks more weight. This can be interpreted as probing whether the signal is \textbf{continuous} (stronger signal $\Rightarrow$ stronger expected return) rather than a pure decile membership effect.
\end{itemize}
Empirically, comparing the three series in Figure~\ref{fig:q1_longshort_series} helps identify whether the momentum premium is primarily coming from (i) smaller names, (ii) bigger names, or (iii) the most extreme momentum observations within each decile.


% --------------------------Question 2----------------------------------
\newpage
\section{Question 2: CAPM $\alpha$ and $\beta$ for the 10 portfolios}
\subsection{Method}
For each decile portfolio $p$, we run the CAPM regression on \textbf{excess returns}:
\[
(R_{p,t}-R_{f,t}) = \alpha_p + \beta_p (R_{m,t}-R_{f,t}) + \varepsilon_{p,t}.
\]
Standard errors are the OLS standard errors (as in the notebook).

\subsection{Results}
\begin{table}[h!]
\centering
\caption{CAPM results by decile (EW portfolios). Returns and factors are in percent per month.}
\label{tab:q2_capm_ew}
\begin{tabular}{lrrrrrr}
\toprule
decile & alpha & beta & se\_alpha & se\_beta & t\_alpha & t\_beta \\
\midrule
1 & -0.7166 & 1.4948 & 0.2328 & 0.0526 & -3.0790 & 28.4428 \\
2 & -0.2223 & 1.1569 & 0.1403 & 0.0317 & -1.5852 & 36.5277 \\
3 & -0.0432 & 0.9992 & 0.1033 & 0.0233 & -0.4185 & 42.8502 \\
4 & -0.0395 & 0.9352 & 0.0817 & 0.0185 & -0.4834 & 50.6866 \\
5 & 0.0985 & 0.8873 & 0.0704 & 0.0159 & 1.4000 & 55.8437 \\
6 & 0.1399 & 0.8878 & 0.0689 & 0.0156 & 2.0301 & 57.0722 \\
7 & 0.1390 & 0.8927 & 0.0783 & 0.0177 & 1.7751 & 50.4771 \\
8 & 0.1361 & 0.9293 & 0.1038 & 0.0234 & 1.3115 & 39.6549 \\
9 & 0.1263 & 1.0099 & 0.1500 & 0.0339 & 0.8419 & 29.8099 \\
10 & 0.2716 & 1.2142 & 0.2466 & 0.0557 & 1.1013 & 21.8070 \\
\bottomrule
\end{tabular}

\end{table}

\begin{table}[h!]
\centering
\caption{CAPM results by decile (VW portfolios). Returns and factors are in percent per month.}
\label{tab:q2_capm_vw}
\small
{\begin{tabular}{lrrrrrr}
\toprule
decile & alpha & beta & se\_alpha & se\_beta & t\_alpha & t\_beta \\
\midrule
1 & -0.7089 & 1.4879 & 0.2305 & 0.0521 & -3.0754 & 28.5496 \\
2 & -0.2261 & 1.1530 & 0.1394 & 0.0315 & -1.6219 & 36.5824 \\
3 & -0.0426 & 0.9958 & 0.1022 & 0.0231 & -0.4166 & 43.1062 \\
4 & -0.0387 & 0.9326 & 0.0805 & 0.0182 & -0.4814 & 51.2651 \\
5 & 0.0917 & 0.8830 & 0.0692 & 0.0157 & 1.3241 & 56.3987 \\
6 & 0.1351 & 0.8846 & 0.0678 & 0.0153 & 1.9923 & 57.7092 \\
7 & 0.1416 & 0.8880 & 0.0763 & 0.0172 & 1.8559 & 51.4791 \\
8 & 0.1294 & 0.9225 & 0.1015 & 0.0229 & 1.2750 & 40.1985 \\
9 & 0.1267 & 1.0019 & 0.1480 & 0.0335 & 0.8559 & 29.9308 \\
10 & 0.2651 & 1.1999 & 0.2422 & 0.0548 & 1.0944 & 21.9092 \\
\bottomrule
\end{tabular}
}
\end{table}

\begin{table}[h!]
\centering
\caption{CAPM results by decile (RW portfolios). Returns and factors are in percent per month.}
\label{tab:q2_capm_rw}
\begin{tabular}{lrrrrrr}
\toprule
decile & alpha & beta & se\_alpha & se\_beta & t\_alpha & t\_beta \\
\midrule
1 & -0.4880 & 1.3838 & 0.2065 & 0.0467 & -2.3629 & 29.6347 \\
2 & -0.2054 & 1.1419 & 0.1374 & 0.0311 & -1.4944 & 36.7487 \\
3 & -0.0444 & 0.9980 & 0.1029 & 0.0233 & -0.4317 & 42.9112 \\
4 & -0.0382 & 0.9335 & 0.0813 & 0.0184 & -0.4697 & 50.7748 \\
5 & 0.1045 & 0.8832 & 0.0697 & 0.0158 & 1.4992 & 56.0304 \\
6 & 0.1379 & 0.8855 & 0.0684 & 0.0155 & 2.0165 & 57.2629 \\
7 & 0.1409 & 0.8909 & 0.0779 & 0.0176 & 1.8087 & 50.5744 \\
8 & 0.1326 & 0.9248 & 0.1028 & 0.0233 & 1.2898 & 39.7741 \\
9 & 0.1240 & 1.0046 & 0.1497 & 0.0338 & 0.8284 & 29.6799 \\
10 & 0.2683 & 1.2066 & 0.2445 & 0.0553 & 1.0972 & 21.8273 \\
\bottomrule
\end{tabular}

\end{table}

\subsection{Analysis}
\textbf{How to read these tables.} The CAPM $\beta$ describes the market exposure of each decile; the CAPM $\alpha$ is the average excess return not explained by market risk alone.

\textbf{What the results suggest.}
Across the three weighting schemes, the key comparison is whether the \emph{alpha spread} between low- and high-momentum deciles is robust:
\begin{itemize}
  \item If high-momentum deciles have higher CAPM alphas than low-momentum deciles, then momentum cannot be explained away by market exposure alone.
  \item If alphas change materially between EW and VW, it indicates that firm size/weighting inside deciles matters for the abnormal performance.
  \item If RW produces alphas closer to the EW case (or amplifies the spread), that supports the interpretation that \emph{signal strength within a decile} carries incremental information.
\end{itemize}



% --------------------------Question 3----------------------------------
\newpage
\section{Question 3: Long--short momentum portfolio (decile 10 minus decile 1)}
\subsection{Method}
Construct a long--short strategy that is long the top momentum decile and short the bottom momentum decile, held for 1 month and rebalanced monthly. For each weighting scheme we compute:
\begin{itemize}
  \item Average raw return (annualized).
  \item CAPM $\alpha$ and its $t$-statistic.
  \item FF3 $\alpha$ and its $t$-statistic.
  \item Annual Sharpe ratio, using monthly excess returns and scaling by $\sqrt{12}$.
\end{itemize}

\subsection{Results}
\begin{table}[h!]
\centering
\caption{Long--short (10--1) strategy summary statistics (EW/VW/RW).}
\label{tab:q3_ls}
\begin{tabular}{lrrrrrr}
\toprule
 & Avg raw (ann., \%) & CAPM $\alpha$ & $t$ (CAPM $\alpha$) & FF3 $\alpha$ & $t$ (FF3 $\alpha$) & Sharpe (ann.) \\
\midrule
Equal-Weighted & 9.3647 & 0.5963 & 1.4600 & 0.6154 & 1.5047 & 0.2510 \\
Value-Weighted & 9.1353 & 0.5759 & 1.4235 & 0.5946 & 1.4680 & 0.2448 \\
Rank-Weighted & 7.5043 & 0.4320 & 1.1102 & 0.4517 & 1.1600 & 0.1909 \\
\bottomrule
\end{tabular}

\end{table}

\subsection{Analysis}
Table~\ref{tab:q3_ls} shows that the long--short strategy earns a positive average return under all three weighting schemes, with EW and VW fairly close and RW somewhat smaller in this run.

\textbf{Risk-adjusted performance.}
Both CAPM and FF3 alphas are positive, which is consistent with momentum being more than just market exposure.
However, the $t$-statistics on these alphas are around the 1--1.5 range here, which is \textbf{not conventionally significant} at common thresholds (e.g., $|t|\ge 1.96$ for 5\%). Interpreting this as a student exercise: the point estimates suggest a momentum premium, but the statistical evidence is modest in this particular sample and specification.

\textbf{Why EW $\approx$ VW but RW is smaller.}
EW and VW being close suggests the premium is not driven solely by either small names or the largest names in this top-1000 universe. RW being smaller can happen when:
\begin{itemize}
  \item the signal is informative mainly at the decile level (membership), not strongly increasing within-decile, or
  \item emphasizing the extremes increases idiosyncratic volatility (hurting Sharpe), or
  \item the rank-weighting amplifies sensitivity to outliers in the momentum signal.
\end{itemize}


% --------------------------Question 4----------------------------------
\newpage
\section{Question 4: $\tau$-month horizon momentum strategy}
\subsection{Economic prediction}
\textbf{Underreaction story.} If momentum is driven by investors \emph{under-reacting} to news, then the momentum effect should \textbf{persist for some horizons}: FF $\alpha(\tau)$ should remain positive for small-to-medium $\tau$ and only decay gradually as prices incorporate information.

\textbf{Overreaction story.} If momentum is driven by \emph{over-reaction}, then one expects \textbf{reversal at longer horizons}: FF $\alpha(\tau)$ may become smaller and eventually negative as mispricing corrects.

\subsection{Empirical results}
\begin{table}[h!]
\centering
\caption{Representative horizons: FF3 $\alpha$ and $t$-stat for the $\tau$-month horizon long--short strategy.}
\label{tab:q4_small}
\begin{tabular}{lrrrr}
\toprule
$\tau$ & EW $\alpha$ & EW t-stats & VW $\alpha$ & VW t-stats \\
\midrule
1 & 0.6154 & 1.5047 & 0.5946 & 1.4680 \\
3 & 0.2819 & 0.7539 & 0.2545 & 0.6942 \\
6 & 0.1056 & 0.3075 & 0.0691 & 0.2096 \\
12 & -0.3446 & -1.0857 & -0.3452 & -1.1921 \\
24 & -0.5994 & -2.2358 & -0.5138 & -2.2861 \\
36 & -0.3051 & -1.2068 & -0.2175 & -1.1093 \\
\bottomrule
\end{tabular}

\end{table}

\begin{figure}[h!]
\centering
\includegraphics[width=0.92\textwidth]{hw1_report_assets/q4_horizon_alpha_tstat.png}
\caption{FF3 $\alpha$ and $t$-statistics for the $\tau$-month horizon momentum strategy as a function of $\tau$ (EW and VW).}
\label{fig:q4_horizon}
\end{figure}

\subsection{Conclusion}
Figure~\ref{fig:q4_horizon} summarizes how the risk-adjusted momentum premium evolves as the holding horizon increases. The key takeaway is whether $\alpha(\tau)$ remains positive (supporting underreaction/persistence) or eventually reverses (supporting overreaction/correction). The associated $t$-statistics indicate how strongly the data support those patterns at each horizon.


% --------------------------Appendix----------------------------------
\appendix
\section{Appendix A: Code}
The full code used to produce these results is included in the repository as a generated Python script extracted from the notebook. See:
\begin{center}
\texttt{hw1/code/hw1\_appendix.py}
\end{center}

\lstinputlisting{../code/hw1_appendix.py}

\end{document}

